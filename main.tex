%CS-113 S18 HW-2
%Released: 2-Feb-2018
%Deadline: 16-Feb-2018 7.00 pm
%Authors: Abdullah Zafar, Emad bin Abid, Moonis Rashid, Abdul Rafay Mehboob, Waqar Saleem.


\documentclass[addpoints]{exam}
\usepackage{amsmath}
\usepackage{mathtools}
\DeclarePairedDelimiter{\ceil}{\lceil}{\rceil}
% Header and footer.
\pagestyle{headandfoot}
\runningheadrule
\runningfootrule
\runningheader{CS 113 Discrete Mathematics}{Homework II}{Spring 2018}
\runningfooter{}{Page \thepage\ of \numpages}{}
\firstpageheader{}{}{}

\boxedpoints
\printanswers
\usepackage[table]{xcolor}
\usepackage{amsfonts,graphicx,amsmath,hyperref}
\title{Habib University\\CS-113 Discrete Mathematics\\Spring 2018\\HW 2}
\author{$sm04120$}  % replace with your ID, e.g. oy02945
\date{Due: 19h, 16th February, 2018}


\begin{document}
\maketitle

\begin{questions}



\question

%Short Questions (25)

\begin{parts}

 
  \part[5] Determine the domain, codomain and set of values for the following function to be 
  \begin{subparts}
  \subpart Partial
  \subpart Total
  \end{subparts}

  \begin{center}
    $y=\sqrt{x}$
  \end{center}

  \begin{solution}
    % Write your solution here
  \begin{subparts}
  \subpart If we consider the domain and co domain to be $\mathbb{R}$, here y would undefined for negative real numbers and the function would be partial for non-negative integers

  \subpart If the domain is all non-negative integers and the co domain is all real numbers, the function would be real for all values of x making the function a total function for this domain
  \end{subparts}
  \end{solution}
  
  \part[5] Explain whether $f$ is a function from the set of all bit strings to the set of integers if $f(S)$ is the smallest $i \in \mathbb{Z}$� such that the $i$th bit of S is 1 and $f(S) = 0$ when S is the empty string. 
  
  \begin{solution}
    % Write your solution here
    If we consider a bit string with only zeros, as the bit string has no ones and is not an empty string, $f(s)$ will be undefined hence it is not a function
  \end{solution}

  \part[15] For $X,Y \in S$, explain why (or why not) the following define an equivalence relation on $S$:
  \begin{subparts}
    \subpart ``$X$ and $Y$ have been in class together"
    \subpart ``$X$ and $Y$ rhyme"
    \subpart ``$X$ is a subset of $Y$"
  \end{subparts}

  \begin{solution}
    % Write your solution here
    \begin{subparts}
    \subpart If X and Y have been in class together and X=Y, X has been in a class with himself showing that this is a reflexive relation. If X and Y have been in class together that means Y and X have been in class together showing that this is a symmetric relation. If X and Y have been in class together, and Y and Z have been in class together, that does not imply that X and Z have been in class together. Since this is not a transitive relation, this is not an equivalence relation.
    \subpart If X and Y rhyme, and X=Y then X and X rhyme hence showing that this relation is reflexive. If X and Y rhyme, Y and X can rhyme showing that this is a symmetric relation. If X and Y rhyme and Y and Z rhyme than that means X and Z rhyme showing that this is a transitive relation. Hence this is an equivalence relation.
    \subpart If X is a subset of X and Y=X, then X is a subset of X showing that this is an reflexive relation. If X is a subset of Y and Z is a subset of X, then that means Z is a subset of Y showing that is is a transitive relation. If X is a subset of Y, Y cannot be a subset of X, if Y and X are not equal, showing that this is not a symmetric relation hence this is not an equivalence relation.
  \end{subparts}
  \end{solution}

\end{parts}

%Long questions (75)
\question[15] Let $A = f^{-1}(B)$. Prove that $f(A) \subseteq B$.
  \begin{solution}
    % Write your solution here
    As an inverse of f exists, f is bijective. Since f is bijective, A and B are bijective which means f(A) can never be a proper subset of B, rather it can be an improper subset.
    
  \end{solution}

\question[15] Consider $[n] = \{1,2,3,...,n\}$ where $n \in \mathbb{N}$. Let $A$ be the set of subsets of $[n]$ that have even size, and let $B$ be the set of subsets of $[n]$ that have odd size. Establish a bijection from $A$ to $B$, thereby proving $|A| = |B|$. (Such a bijection is suggested below for $n = 3$) 

\begin{center}

  \begin{tabular}{ |c || c | c | c |c |}
    \hline
 A & $\emptyset$ & $\{2,3\}$ & $\{1,3\}$ & $\{1,2\}$ \\ \hline
 B & $\{3\}$ & $\{2\}$ & $\{1\}$ & $\{1,2,3\}$\\\hline
\end{tabular}
\end{center}

  \begin{solution}
    % Write your solution here
    We first construct a function f(x) in which y is a random element from the set n,
    \[ f(x)=
    \begin{cases} 
      x-\{y\} &  \textrm{if y exists in x}\\
      x \cup \{y\} & \textrm{if y does not exist in x}  \\
   \end{cases}
    \]
    Here,  if x is a set with even elements, it will output a set with an even number of elements. Since we are either adding or removing from the input set, we always get one output showing us that is function so injective.
    
    Since the output is always odd, if we go in reverse and add or subtract an element, we always get a set with even number of elements which is our domain. This means that each image has a pre image showing that this function is surjective.
    
    Since the function is both injective and surjective, it is bijective and since it is bijective, we have proved that the domain and co domain have the same cardinality.
  \end{solution}
  
\question Mushrooms play a vital role in the biosphere of our planet. They also have recreational uses, such as in understanding the mathematical series below. A mushroom number, $M_n$, is a figurate number that can be represented in the form of a mushroom shaped grid of points, such that the number of points is the mushroom number. A mushroom consists of a stem and cap, while its height is the combined height of the two parts. Here is $M_5=23$:

\begin{figure}[h]
  \centering
  \includegraphics[scale=1.0]{m5_figurate.png}
  \caption{Representation of $M_5$ mushroom}
  \label{fig:mushroom_anatomy}
\end{figure}

We can draw the mushroom that represents $M_{n+1}$ recursively, for $n \geq 1$:
\[ 
    M_{n+1}=
    \begin{cases} 
      f(\textrm{Cap\_width}(M_n) + 1, \textrm{Stem\_height}(M_n) + 1, \textrm{Cap\_height}(M_n))  & n \textrm{ is even} \\
      f(\textrm{Cap\_width}(M_n) + 1, \textrm{Stem\_height}(M_n) + 1, \textrm{Cap\_height}(M_n)+1) & n \textrm{ is odd}  \\      
   \end{cases}
\]

Study the first five mushrooms carefully and make sure you can draw subsequent ones using the recurrence above.

\begin{figure}[h]
  \centering
  \includegraphics{mushroom_series.png}
  \caption{Representation of $M_1,M_2,M_3,M_4,M_5$ mushrooms}
  \label{fig:mushroom_anatomy}
\end{figure}

  \begin{parts}
    \part[15] Derive a closed-form for $M_n$ in terms of $n$.
  \begin{solution}
    % Write your solution here
   As we can see, the sequence of cap widths is 2,3,4,5,6...
    
    \[ Cap\_width(m_n)=n+1 \]
    
    As we can see, the sequence of cap heights is 1,2,2,3,3,3,4,4,4,4,....
    \[ Cap\_height(M_n)=  \ceil{(1+n)/2} \]
    
    As we can see, the sequence of stem heights is 0,1,2,3,4,....
    
    \[ Stem\_height(M_n)= n-1 \]
    
    The number dots in the stem is equal to the stem height multiplied by 2 and so is 2*(n-1).
    
    The number of dots in the cap is is the sum of an arithmetic progression where the starting value is the cup width, the difference is -1, and the last term is the cup height. As the sum of an arithmetic progression is (2a+(n-1)(d))/2, the number of dots in the cap is $(2(n+1)+(\ceil{(1+n)/2}-1)(-1))*\ceil{(1+n)/2}/2$
    
    Thus, $M_n=(2(n+1)+(\ceil{(1+n)/2}-1)(-1))*\ceil{(1+n)/2}/2+ 2*(n-1)$

    
    
  \end{solution}
    \part[5] What is the total height of the $20$th mushroom in the series? 
  \begin{solution}
    % Write your solution here
    Total height=Stem height+Cup height
    
    $Stem\_height=n-1=20-1=19$
    
    $Cap\_height=\ceil{(1+n)/2}=\ceil{(1+20)/2}=\ceil{(21)/2}=\ceil{(10.5)}=11$
    
    $11+19=30$
  \end{solution}
\end{parts}

\question
    The \href{https://en.wikipedia.org/wiki/Fibonacci_number}{Fibonacci series} is an infinite sequence of integers, starting with $1$ and $2$ and defined recursively after that, for the $n$th term in the array, as $F(n) = F(n-1) + F(n-2)$. In this problem, we will count an interesting set derived from the Fibonacci recurrence.
    
The \href{http://www.maths.surrey.ac.uk/hosted-sites/R.Knott/Fibonacci/fibGen.html#section6.2}{Wythoff array} is an infinite 2D-array of integers where the $n$th row is formed from the Fibonnaci recurrence using starting numbers $n$ and $\left \lfloor{\phi\cdot (n+1)}\right \rfloor$ where $n \in \mathbb{N}$ and $\phi$ is the \href{https://en.wikipedia.org/wiki/Golden_ratio}{golden ratio} $1.618$ (3 sf).

\begin{center}
\begin{tabular}{c c c c c c c c}
 \cellcolor{blue!25}1 & 2 & 3 & 5 & 8 & 13 & 21 & $\cdots$\\
 4 & \cellcolor{blue!25}7 & 11 & 18 & 29 & 47 & 76 & $\cdots$\\
 6 & 10 & \cellcolor{blue!25}16 & 26 & 42 & 68 & 110 & $\cdots$\\
 9 & 15 & 24 & \cellcolor{blue!25}39 & 63 & 102 & 165 & $\cdots$ \\
 12 & 20 & 32 & 52 & \cellcolor{blue!25}84 & 136 & 220 & $\cdots$ \\
 14 & 23 & 37 & 60 & 97 & \cellcolor{blue!25}157 & 254 & $\cdots$\\
 17 & 28 & 45 & 73 & 118 & 191 & \cellcolor{blue!25}309 & $\cdots$\\
 $\vdots$ & $\vdots$ & $\vdots$ & $\vdots$ & $\vdots$ & $\vdots$ & $\vdots$ & \color{blue}$\ddots$\\
 

\end{tabular}
\end{center}

\begin{parts}
  \part[10] To begin, prove that the Fibonacci series is countable.
 
    \begin{solution}
    % Write your solution here
    If we create a function f(x) where $ x \in \mathbb{Z^+}$,
    \[ f(x)=
    \begin{cases} 
      1 &  \textrm{if x=1}\\
      2 & \textrm{if x=2}  \\
      f(x-1)+f(x-2) & \textrm{x$>$2} 
   \end{cases}
    \]
    Through this function, we can see that all positive integers are being mapped to the Fibonacci series. Since each integer maps to a unique element in the Fibonacci, the function is injective. Since each element in the Fibonacci is being mapped from a unique integer, the function is surjective therefore the function is bijective and since we have proved a bijection from the set of positive integers to the Fibonacci series, we have proved that the Fibonacci series is countable.
  \end{solution}
  \part[15] Consider the Modified Wythoff as any array derived from the original, where each entry of the leading diagonal (marked in blue) of the original 2D-Array is replaced with an integer that does not occur in that row. Prove that the Modified Wythoff Array is countable. 

  \begin{solution}
    % Write your solution here
    Every row in the Wythoff array is a Fibonnaci sequence. As we have proved in part (a) that the Fibonacci series is countable, every row of the Wythoff is countable, as every row is a modified Fibonacci series. If we replace each leading diagonal in each row, we will be replacing each with a number not present in the row, meaning every element in the row will be a unique number. Because of this, each modified row will still remain countable and bijective with $\mathbb{Z^+}$
    
    If we take the cartesian product of $\mathbb{Z^+}$, $\mathbb{Z^+} \times \mathbb{Z^+}$, we can see each tuple in $\mathbb{Z^+} \times \mathbb{Z^+}$ as (a,b). a can be seen as the row number and b can be seen as the column number. With this we can establish a bijection between $\mathbb{Z^+} \times \mathbb{Z^+}$ and the Wythoff array similiar to how we did it in part (a). As each element of the Wythoff array has a different column number and row number, their is a surjection between $\mathbb{Z^+} \times \mathbb{Z^+}$ and the Wythoff. As we have already established that each row will have no repeating elements, if an element does repeat in the Wythoff, it will be in a different column and hence will be related to a different tuple of $\mathbb{Z^+} \times \mathbb{Z^+}$, thus establishing an injection. As we have established both and injection and surjection, we can say that the Wythoff array is bijective with \textbf{$\mathbb{Z^+} \times \mathbb{Z^+}$}. As $\mathbb{Z^+}$ is countable, $\mathbb{Z^+} \times \mathbb{Z^+}$ is also countable and since we have established a bijection between the Wythoff and $\mathbb{Z^+} \times \mathbb{Z^+}$, we can safely say that the wythoff is countable.
  \end{solution}
\end{parts}

\end{questions}

\end{document}
